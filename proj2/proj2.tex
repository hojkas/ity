\documentclass[a4paper, 11pt, twocolumn]{article}
\usepackage[utf8]{inputenc}
\usepackage[IL2]{fontenc}
\usepackage[czech]{babel}
\usepackage[left=1.5cm,text={18cm, 25cm},top=2.5cm]{geometry}
\usepackage{times}
\usepackage{amsmath, amsthm, amssymb}

\theoremstyle{definition}
\newtheorem{definice}{Definice}

\theoremstyle{definition}
\newtheorem{veta}{Věta}

\begin{document}
\begin{titlepage}
    \begin{center}
        \textsc{\Huge Fakulta informačních technologií\\[3.75mm]Vysoké učení technické v Brně}\\
        \vspace{\stretch{0.382}}
        {\LARGE Typografie a publikování -- 2. projekt\\
        Sazba dokumentů a matematických výrazů\\}
        \vspace{\stretch{0.618}}
        \Large 2019 \hfill name
    \end{center}
\end{titlepage}
\newpage
\section*{Úvod}

V této úloze si vyzkoušíme sazbu titulní strany, matematických vzorců, prostředí a dalších textových struktur obvyklých pro technicky zaměřené texty (například rovnice (\ref{rov1}) nebo Definice \ref{def1} na straně \pageref{def1}). Pro odkazování na vzorce a struktury zásadně používáme příkaz \verb|\label| a \verb|\ref| případně \verb|\pageref| pokud se chceme odkázat na stranu výskytu.

Na titulní straně je využito sázení nadpisu podle optického středu s využitím zlatého řezu. Tento postup byl probírán na přednášce. Dále je použito odřádkování se zadanou relativní velikostí 0.4 em a 0.3 em.

\section{Matematický text}

Nejprve se podíváme na sázení matematických symbolů a výrazů v plynulém textu včetně sazby definic a vět s využitím balíku \verb|amsthm|. Rovněž použijeme poznámku pod čarou s použitím příkazu \verb|\footnote|. Někdy je vhodné použít konstukci \verb|\mbox{}|, která říká, že text nemá být zalomen.

\begin{definice}
    \label{def1} Zásobníkový automat \emph{(ZA) je definován jako sedmice tvaru $A = (Q, \Sigma, \Gamma, \delta, q_0, Z_0, F)$, kde:}

    \begin{itemize}
        \item \emph{Q je konečná množina} vnitřních (řídících) stavů,
        \item \emph{$\Sigma$ je konečná} vstupní abeceda,
        \item \emph{$\Gamma$ je konečná} zásobníková abeceda,
        \item \emph{$\delta$ je} přechodová funkce $Q \times (\Sigma \cup \{\epsilon\}) \times \Gamma \rightarrow 2^{Q \times \Gamma^*}$
        \item \emph{$q_0 \in$ Q je} počáteční stav, \emph{$Z_0 \in \Gamma$ je} startovací symbol zásobníku \emph{a F $\subseteq$ Q je množina} koncových stavů.
    \end{itemize}
\end{definice}

Nechť $P = (Q, \Sigma, \Gamma, \delta, q_0, Z_0, F)$ je zásobníkový automat. \emph{Konfigurací} nazveme trojici $(q, w, \alpha) \in Q \times \Sigma^* \times \Gamma^*$, kde $q$ je aktuální stav vnitřního řízení, $w$ je dosud nezpracovaná část vstupního řetězce a $\alpha = Z_{i_1}, Z_{i_2} \dots Z_{i_k}$ je obsah zásobníku\footnote{$Z_{i_1}$ je vrchol zásobníku}.

\subsection{Podsekce obsahující větu a odkaz}
\begin{definice}
\label{def2}
    Řetězec \emph{w} nad abecedou $\Sigma$ je přijat ZA \emph{A~jestliže $(q_0, w, Z_0$)  $\underset{A}{\overset{*}{\vdash}} (q_F, \epsilon, \gamma)$ pro nějaké $\gamma \in \Gamma^*$ a $q_F \in F$. Množinu $L(A) = \{w\ |\ w$ je přijat ZA $A\} \subseteq \Sigma^*$ nazýváme} jazyk přijímaný TS $M$. 
\end{definice}

Nyní si vyzkoušíme sazbu vět a důkazů opět s použitím balíku \verb|amsthm|.

\begin{veta}
\label{vet1}
    \emph{Třída jazyků, které jsou přijímány ZA, odpovídá} bezkontextovým jazykům.
\end{veta}

\begin{proof}
\label{duk1}
    V důkaze vyjdeme z Definice \ref{def1} a \ref{def1}.
\end{proof}

\section{Rovnice a odkazy}

Složitější matematické formulace sázíme mimo plynulý text. Lze umístit několik výrazů na jeden řádek, ale pak je třeba tyto vhodně oddělit, například příkazem \verb|\quad|.

\begin{center}
$\sqrt[i]{x^3_i}$ kde $x_i$ je $i$-té sudé číslo splňující $x_i^{2-x_i^2} \leq x_i^{y_i^3}$
\end{center}

V rovnici (\ref{rov1}) jsou využity tři typy závorek s různou explicitně definovanou velikostí.

\begin{eqnarray}
\label{rov1}
    x&=&\bigg[\Big\{\big[a+b\big]*c\Big\}^d\ominus1\bigg]^{1/2}\\
    y&=&\lim_{x\to\infty}\frac{\frac{1}{\log_{10}x}}{\sin^2x+\cos^2x} \nonumber
\end{eqnarray}

V této větě vidíme, jak vypadá implicitní vysázení limity $\lim_{n\to\infty}f(n)$ v normálním odstavci textu. Podobně je to i s dalšími symboly jako $\prod_{i=1}^n 2^i$ či $\bigcap_{A\in \mathcal{B}}A$. V případě vzorců  $\lim\limits_{n\to\infty}f(n)$ a $\prod\limits_{i=1}^n 2^i$ jsme si vynutili méně úspornou sazbu příkazem \verb|\limits|.

\begin{eqnarray}
    \int_b^a g(x)dx &= &-\int\limits_a^b f(x)dx\\
    \overline{\overline{A\wedge B}} &\Leftrightarrow &\overline{\overline{A} \vee \overline{B}}
\end{eqnarray}

\section{Matice}

Pro sázení matic se velmi často používá prostředí \verb|array| a závorky (\verb|\left|, \verb|\right|).

$$\left[\begin{array}{ccc}
     &\widehat{\beta + \gamma}&\hat{\pi}\\
    \vec{a}&\overleftrightarrow{AC}&\\
\end{array} \right] = 1 \Longleftrightarrow \mathbb{Q} = \mathbf{R}$$

$$\mathbf{A} = \left|\begin{array}{cccc}
     a_{11}&a_{12}&\dots&a_{1n}\\
     a_{21}&a_{22}&\dots&a_{2n}\\
     \vdots&\vdots&\ddots&\vdots\\
     a_{m1}&a_{m2}&\dots&a_{mn}\\
\end{array} \right| = \begin{array}{cc}
     t&u\\
     u&w\\ 
\end{array} = tw - uv$$

Prostředí \verb|array| lze úspěšně využít i jinde.

$$
\left(\begin{array}{c}
     n\\
     k\\ 
\end{array} \right) =
\begin{cases}
    0&$pro $k<0$ nebo $k>n\\
    \frac{n!}{k!(n-k)!}&$pro $0\leq k \leq n\\
\end{cases}$$


\end{document}
