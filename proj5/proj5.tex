\documentclass[10pt]{beamer}
\usepackage[utf8]{inputenc}
\usepackage[IL2]{fontenc}
\usepackage{times}
\usepackage{graphics}
\usepackage{listings}
\usepackage{color}
\usepackage[czech]{babel}
\usepackage{url}
 
\definecolor{codegreen}{rgb}{0,0.6,0}

\lstdefinestyle{code}{
    basicstyle=\ttfamily,
    commentstyle=\color{codegreen},
    keywordstyle=\color{blue},
    numberstyle=\tiny\color{gray},
    breakatwhitespace=false,         
    keepspaces=true,                 
    numbers=left,                    
    numbersep=15pt,                  
    tabsize=4
}
\lstset{style=code}

\usetheme{Copenhagen}
\setbeamertemplate{}[frame number]

\title{Třídíci algoritmy}
\subtitle{Bubble Sort}
\author{xplagi0b}
\date{\today}

\begin{document}
\maketitle
\begin{frame}{Třídící (řadící) algoritmus}
    \begin{itemize}
    \item Třídící algoritmus zajišťuje uspořádání daných dat podle požadovaného pořadí
    \item Řadí na základě určené části dat ("klíče")
    \item Nejčastěji jde o numerické či abecední seřazení
    \pause
    \item Na vstupu algoritmu je posloupnost záznamů:
    $$S = S_1, S_2, \dots, S_n$$
    \pause
        \item A výstupem je:
        
        \begin{itemize}
            \item Posloupnost je seřazená:
            $$S'_1 \leq S'_2 \leq \dots \leq S'_n$$
            \item Posloupnost je permutací původní posloupnosti $S$ (pořadí se mohlo změnit, avšak data zůstavají stejná)
        \end{itemize}
    \end{itemize}
\end{frame}

\begin{frame}{Složitost algoritmu}
    \begin{itemize}
        \item Neexistuje třídící algoritmus dokonalý pro všechna použití
        \item Pro kokrétní použití se volí vhodný algoritmus na základě jeho vlastností, především:
        \begin{itemize}
            \item Časová složitost
            \item Prostorová složitost
            \item Náročnost implementace
            \item \dots
        \end{itemize}
    \end{itemize}
\end{frame}

\begin{frame}{Bubble Sort}
    \begin{itemize}
        \item též \textbf{řazení záměnou} je algoritmus opakovaně procházející seznam, kde při každém projití porovnává sousedící prvky a případě potřeby je prohodí
        \item implementačně jednochý algoritmus s časovou náročností $O(n^2)$, kvůli které se v praxi příliš nepoužívá
        \pause
        \item Postup algoritmu pro 3 prvky:
    \end{itemize}
    
    \begin{figure}[ht]
        \centering
        \setlength{\unitlength}{0.5cm}
        \begin{picture}(15,6.5)
            \put(0,6){\line(1,0){3}}
            \put(0,6){\line(0,-1){1}}
            \put(1,6){\line(0,-1){1}}
            \put(2,6){\line(0,-1){1}}
            \put(3,6){\line(0,-1){1}}
            \put(0,5){\line(1,0){3}}
            
            \put(0.3,5.3){\color{red}8}
            \put(1.3,5.3){\color{red}4}
            \put(2.3,5.3){2}
            \put(4,5.3){Nalezeno špatné pořadí 8 a 4, prohazuje}
            
            \pause
            
            \put(0,4.5){\line(1,0){3}}
            \put(0,4.5){\line(0,-1){1}}
            \put(1,4.5){\line(0,-1){1}}
            \put(2,4.5){\line(0,-1){1}}
            \put(3,4.5){\line(0,-1){1}}
            \put(0,3.5){\line(1,0){3}}
            
            \put(0.3,3.8){4}
            \put(1.3,3.8){\color{red}8}
            \put(2.3,3.8){\color{red}2}
            \put(4,3.8){Nalezeno špatné pořadí 8 a 2, prohazuje}
            \pause
            
            \put(0,3){\line(1,0){3}}
            \put(0,3){\line(0,-1){1}}
            \put(1,3){\line(0,-1){1}}
            \put(2,3){\line(0,-1){1}}
            \put(3,3){\line(0,-1){1}}
            \put(0,2){\line(1,0){3}}
            
            \put(0.3,2.3){\color{red}4}
            \put(1.3,2.3){\color{red}2}
            \put(2.3,2.3){8}
            \put(4,2.3){Nalezeno špatné pořadí 4 a 2, prohazuje}
            \pause
            
            \put(0,1.5){\line(1,0){3}}
            \put(0,1.5){\line(0,-1){1}}
            \put(1,1.5){\line(0,-1){1}}
            \put(2,1.5){\line(0,-1){1}}
            \put(3,1.5){\line(0,-1){1}}
            \put(0,0.5){\line(1,0){3}}
            
            \put(0.3,0.8){2}
            \put(1.3,0.8){4}
            \put(2.3,0.8){8}
            \put(4,0.8){Seřazená posloupnost}
        \end{picture}
    \end{figure}
\end{frame}

\begin{frame}{Příklad implementace}
    \lstinputlisting[language=C]{codeexample.c}
\end{frame}

\begin{frame}{Zdroje}
    \nocite{*}
    \bibliographystyle{czechiso}
    \bibliography{proj5}
\end{frame}

\end{document}
