\documentclass[a4paper, 11pt]{article}
\usepackage[utf8]{inputenc}
\usepackage[IL2]{fontenc}
\usepackage[czech]{babel}
\usepackage[left=2cm,text={17cm, 24cm},top=3cm]{geometry}
\usepackage{times}
\usepackage{url}
\usepackage{breakurl}
\usepackage[breaklinks]{hyperref}
\def\UrlBreaks{\do\/\do-}

\begin{document}

\begin{titlepage}
    \begin{center}
        \textsc{\Huge Vysoké učení technické v Brně \\[3.5mm] \huge Fakulta informačních technologií}\\
        \vspace{\stretch{0.382}}
        {\LARGE Typografie a publikování -- 4 projekt\\
        \huge Citace\\}
        \vspace{\stretch{0.618}}
        \Large \today \hfill your\_name
    \end{center}
\end{titlepage}

\section{Sazba}

\begin{quote}
    Sazba (od \uv{sázeti}) je původně tisková forma pro tisk z výšky (knihtisk), tvořená jednotlivými literami, případně obrázky. V širším slova smyslu hotová předloha pro tisk, zejména textová. \cite{sazbaWiki}
\end{quote}


Podle způsobu vyhotovení se dále dělí na ruční, strojovou, elektronickou sazbu a další.

Ruční sazba spočívá skládání jednotlivých kovových liter k vysázení do stroje a poté opětovnému ručnímu rozebrání. Jako forma tisku se objevila nejdříve, avšak kvůli časové náročnosti byla s pokrokem nahrazena. Strojová sazba již spoléhala na skládání jednotlivých řádků strojem. Nejmodernější elektronická sazba přišla s nástupem výpočetní techniky. \cite{Ottova2003} \cite{sazbaWiki}

\subsection{Knihtisk}
    Knihtisk je považován za jeden z nejvýznamějších vynálezů. Ovlivnil jak vývoj lidské společnosti, tak i prakticky všechna odvětví naší činnosti. Přinesl způsob šíření vědomostí a vzdělanosti na dosud nevídané úrovni a v zdokonalené formě je využíván dodnes. \cite{Otazniky2014} \cite{ABCknihtisk} \cite{Mensik2013}

    První tištěná kniha se objevila v Číně již r. 868 s použítím techniky tisku z desky. V Evropě byl knihtisk objeven J. Gutenbergem nejspíše roku 1445. Knihtisk se nejprve rozšířil do dalších německých měst, brzy však i po celé Evropě. Jeho rozmach pomohl překlenout přechod ze středověku do renesance a stále je významnou součástí naší společnosti. \cite{Ottova2003} \cite{Steinberg1996} \cite{Voit2008}

\subsection{Typografie}
Nad sazbou a grafickou úpravou tiskovin se postupně vyvinul celý obor -- typografie. V rámci typografie je zkoumáno, jak vhodně předávat informace prostřednictvím tištěných či elektronických materiálů. Důraz je kladen na přehlednost, účelnost a estetickou stránku.
\cite{OlsakTypo}

Zatímco v minulosti patřilo zabývání se sazbou pouze odborníkům, dnes může v té či oné formě publikovat téměř každý. Softwarové nástroje typu WYSIWYG (What You See Is What You Get), jako je například MS Word, nebo editory na jiném principu, např. \LaTeX, však za uživatele nevyřeší vše. Ovládat zásadní typografické zásady by ideálně mělo být vlastní každému uživateli. Bohužel v praxi se najde mnoho lidí, kterým tato pravidla nic neříkají. \cite{Olsak1996} \cite{Sirucek2006}

\newpage
\bibliographystyle{czechiso}
\bibliography{proj4}

\end{document}
